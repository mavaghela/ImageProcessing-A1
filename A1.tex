\documentclass{article}
\usepackage[utf8]{inputenc}
\usepackage {tikz}

\title{CSC420 A1}
\author{Mansi Vaghela}
\date{January 2017}

\begin{document}

\maketitle

\section*{Question 1}

\subsection*{a.}

\texttt{function out = q1a(I, origF)
\newline\indent
  F = rot90(origF);
  \newline\indent
  F = rot90(F);
  \newline\indent
  Image = imread(I);
  \newline\indent
  imageSize = size(Image);
  \newline\indent
  filterSize = size(F);
  \newline\indent
  result = zeros(imageSize(1), imageSize(2));
  \newline\indent
  padded = padarray(Image, [1 1]);
  \newline\indent
  for n = 1:imageSize(1);
  \newline\indent\indent
    for m = 1:imageSize(2);
    \newline\indent\indent\indent
      val = convolution(n, m, F, padded);
      \newline\indent\indent\indent
      result(n, m) = val;
      \newline\indent\indent
    end
    \newline\indent
  end
  \newline\indent
  out = result
  \newline
end
\newline\newline
function out = convolution(n, m, F, padded)
\newline\indent
  filterSize = size(F);
  \newline\indent
  val = 0;
  \newline\indent
  for k = 1:filterSize(1)
  \newline\indent\indent
  tempm = m;
  \newline\indent\indent
    for l = 1:filterSize(2)
    \newline\indent\indent\indent
      val += F(k, l) * padded(n, tempm);
      \newline\indent\indent\indent
      tempm++;
     \newline\indent\indent
     end
     \newline\indent\indent
     n++;
     \newline\indent
  end
  \newline\indent
  out = val;
  \newline
end
}

\subsection*{b.}

Yes it is possible to write a convolution in one pixel as a dot product between two vectors. You would have to convert both the image and filter matrix into a 1D array in the same way and then take the dot product of the two. You cannot write a full convolution via matrix multiplication because the image would need to have the same number of rows in its matrix as the number of columns in the filter.

\section*{Question 2}

\subsection*{a.}
An $m \times m$ filter requires $m^{2}$ operations per pixel. An $n \times n$ image has $n^{2}$ pixels thus the computational cost of a convolution is $O(m^{2} \times n^{2})$. If $h$ is a separable filter then then only $2m$ operations are required per pixel so the computational cost would be $O(2m \times n^{2}) = O(m \times n^{2})$.

\subsection*{b.}
We know that $\sigma  = \sqrt{\sigma^{2}_{1} + \sigma^{2}_{2}}$
\newline\newline
$\sigma = \sqrt{1^{2} + 3^{2}}$ = $\sqrt{10}$

\subsection*{c.}
INCOMPLETE
\texttt{Some code}

\subsection*{d.}
\texttt{\noindent
gfilter = fspecial('gaussian', [18, 18],  10);
\newline
I = imread('image.jpg');
\newline
img = imfilter(I, gfilter, 'conv');
}

\begin{figure}[h!]
\centering
\includegraphics[width=7.5cm, height=8cm]{smooth_image.jpg}
\label{fig:sample graph}
\end{figure}


\newpage
\subsection*{e.}
If you an $x \times y$ filter and an $m \times n$ image, each pixel in the image would require $xy$ operations per pixel to perform a convolution. This means the whole image would require $(x\times y) \times (m \times n) $ operations. If the filter is separable, then the vertical filter is $x \times 1$ and the horizontal is $y \times 1$, so per pixel $x+y$ operations are required. Thus to perform convolution on the entire image it would require $(x + y) \times (m \times n)$ operations.

\subsection*{f.}
If the Gaussian filter is an $m \times n$ matrix, then the vertical filter is $m \times 1$ and the horizontal filter is $1 \ times n$.

\subsection*{g.}
INCOMPLETE

\section*{Question 3}
\subsection*{a.}
\texttt{
\newline\noindent
waldo = imread('waldo.png');
\newline
waldo = rgb2gray(waldo);
\newline
waldoG = imgradient(waldo);
\newline
waldoMaxgrad =  waldoG/max(max(waldoG));
\newline\newline
template = imread('template.png');
\newline
template = rgb2gray(template);
\newline
templateG = imgradient(template);
\newline
templateMaxgrad =  templateG/max(max(templateG));
}

\begin{figure}[h!]
\centering
\includegraphics[width=10cm, height=6cm]{waldo_grad.png}
\label{fig:sample graph}
\end{figure}

\begin{figure}[h!]
\centering
\includegraphics[width=4cm, height=8cm]{template_grad.png}
\label{fig:sample graph}
\end{figure}


\subsection*{b.}

\texttt{\noindent
function out = findWaldo(im, filter)
\newline\indent
\% convert image (and filter) to grayscale
\newline\indent
im\_input = im;
\newline\indent
im = imread(im);
\newline\indent
filter = imread(filter);
\newline\newline\indent
\% normalized cross-correlation
\newline\indent
out = normxcorr2(filter, im);
\newline\newline\indent
\% plot the cross-correlation results
\newline\indent
figure('position', [100,100,size(out,2),size(out,1)]);
\newline\indent
subplot('position',[0,0,1,1]);
\newline\indent
axis off;
\newline\indent
axis equal;
\newline\newline\indent
\% find the peak in response
\newline\indent
[y,x] = find(out == max(out(:)));
\newline\indent
y = y(1) - size(filter, 1) + 1;
\newline\indent
x = x(1) - size(filter, 2) + 1;
\newline\newline\indent
\% plot the detection's bounding box
\newline\indent
figure('position', [300,100,size(im,2),size(im,1)]);
\newline\indent
subplot('position',[0,0,1,1]);
\newline\indent
imshow(im\_input)
\newline\indent
axis off;
\newline\indent
axis equal;
\newline\indent
rectangle('position', [x,y,size(filter,2),size(filter,1)], 'edgecolor',
\newline\indent
[0.1,0.2,1], 'linewidth', 3.5);
\newline
end
}


\begin{figure}[h!]
\centering
\includegraphics[width=10cm, height=6cm]{found_waldo.png}
\label{fig:sample graph}
\end{figure}

\newpage

\section*{Question 4}
\subsection*{a.}
\texttt{\noindent
I = imread('court.jpg');
\newline
I = rgb2gray(I);
\newline
edges = edge(I, 'canny', [0.2 0.25], 1);
\newline
imshow(edges)
}

\begin{figure}[h!]
\centering
\includegraphics[width=10cm, height=7cm]{canny_court.jpg}
\label{fig:sample graph}
\end{figure}


\subsection*{b.}
INCOMPLETE




\end{document}
